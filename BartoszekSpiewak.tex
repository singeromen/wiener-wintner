% ----------------------------------------------------------------

% AMS-LaTeX Paper ************************************************

% **** -----------------------------------------------------------

\documentclass{amsart}

\usepackage{amsmath}

\usepackage{amssymb}

\usepackage{amscd}

\usepackage[mathscr]{eucal}

%\usepackage[active]{srcltx} % SRC Specials: DVI [Inverse] Search

% ----------------------------------------------------------------

\vfuzz2pt % Don't report over-full v-boxes if over-edge is small

\hfuzz2pt % Don't report over-full h-boxes if over-edge is small

% THEOREMS -------------------------------------------------------

\newtheorem{thm}{Theorem}[section]

\newtheorem{cor}[thm]{Corollary}

\newtheorem{lemma}[thm]{Lemma}

\newtheorem{prop}[thm]{Proposition}

\theoremstyle{definition}

\newtheorem{defn}[thm]{Definition}

\theoremstyle{remark}

\newtheorem{rem}[thm]{Remark}

\newtheorem{exa}[thm]{Example}

\numberwithin{equation}{section}

% MATH -----------------------------------------------------------

\newcommand{\bs}{\rightline{$\blacksquare$}}

\newcommand{\norm}[1]{\left\Vert#1\right\Vert}

\newcommand{\abs}[1]{\left\vert#1\right\vert}

\newcommand{\set}[1]{\left\{#1\right\}}

\newcommand{\Real}{\mathbb R}

\newcommand{\eps}{\varepsilon}

\newcommand{\To}{\longrightarrow}

\newcommand{\BX}{\mathbf{B}(X)}

\newcommand{\Hi}{\mathcal{H}}

\newcommand{\Lo}{\mathcal{L}(\mathcal{H})}

\newcommand{\Cj}{\mathcal{C}_{1}}

\newcommand{\St}{\mathcal{S}}

\def\Ro{{\bf R}}

\def\Co{{\bf C}}

\def\Io{{\bf I}}

\def\Tr{{\rm \,Tr\,}}

\def\c{{\check c}}

\def\d{{\check d}}

\def\endpage{\strut\vfill\eject\noindent}

%\newtheorem{lemma}{Lemma}

\newtheorem{theorem}{Theorem}

\newtheorem{proposition}{Proposition}

\newtheorem{corollary}{Corollary}

\def\Km{K^{\rm m}}

\def\Ks{K^{\rm s}}

% ----------------------------------------------------------------

\begin{document}

\title {A note on a Wiener-Wintner theorem for mean \\ ergodic Markov amenable semigroups }

\author {Wojciech Bartoszek$^1$ and Adam \'Spiewak$^2$ }

\address {Department of Mathematics, Gda\'nsk University of

Technology, \newline

 ul. Narutowicza 11/12, 80 233 Gda\'nsk, Poland}

\email{$^1$bartowk@mifgate.mif.pg.gda.pl, corresponding author}

\email{$^2$adspiewak@gmail.com}



%\subjclass{}%

\keywords {amenable Markov semigroups, Wiener-Wintner ergodic theorem.}



\date {9 April 2015}

%\dedicatory{}%

%\commby{}%

% ----------------------------------------------------------------

\begin{abstract}

We prove a Wiener-Wintner ergodic type theorem for a Markov representation

$ \mathcal{S} =  \{ S_g : g\in G \}$ of a right amenable

semitopological semigroup $G$. We assume that $\mathcal{S}$ is mean

ergodic as a semigroup of linear Markov operators acting on

$(C(K), \| \cdot \|_{\sup })$, where $K$ is a fixed Hausdorff, compact space.

The main result of the paper are necessary and sufficient conditions for

mean ergodicity of a distorted  semigroup  $\{ \chi(g)S_g : g\in G \}$,

where $\chi $ is a semigroup character. Such conditions were obtained before under the

additional assumption that $\mathcal{S}$ is uniquely ergodic.

\end{abstract}

\maketitle

% ----------------------------------------------------------------

\pagestyle{myheadings}



\markboth{Bartoszek \& \'Spiewak }{On Wiener-Wintner ergodic theorem}





\section{Introduction}

The paper contributes towards  a recently published  paper [10]

due to M. Schreiber. To avoid redundancy and keep the format of this note

appropriately compact we generally follow definitions and notation from [10].

However for the convenience of the reader we give a brief summary of

the topic we deal with. Given a compact Hausdorff

space $K$ and the complex Banach lattice $C(K)$ of all continuous complex valued

functions on $K$, a linear contraction operator $S : C(K) \to C(K)$ is call (strongly) mean ergodic if

its Cesaro means $\frac{1}{n}\sum_{j=1}^n S^jf $ converge uniformly  on $K$ (i.e. in the

$\sup $ norm $\| \cdot \|$)  to $Qf$. It is well known that the limit

operator $Q$ is a linear projection on

the manifold $Fix(S) = \{ f\in C(K) : Sf = f \}$ of $S$-invariant functions.

The characterization of  mean ergodicity are today a classical part of

operator ergodic theory and can be found in most monographs (cf. [5], [8]).



Let us recall that a linear operator $S : C(K) \to C(K)$ is called Markov if

$Sf\geq 0$ for all (real valued) nonnegative  $f\in C(K)_+$ and $S\mathbf{1} = \mathbf{1}$.

Clearly any Markov operator has norm $1$, in particular it is a contraction.

Given a semitopological semigroup $G$, a (bounded) representation of $G$ on $C(K)$ is

the semigroup of operators $\mathcal{S} = \{ S_g : g\in G \}$ such that

$S_{g_1g_2} = S_{g_2}S_{g_1}$ and $G \ni g \to S_gf \in C(K)$ is norm continuous for every

$f\in C(K)$ and $\sup_{g\in G} \| S_g \| < \infty$. If all $S_g$ are Markovian, then the representation

is called Markovian.

A (complex) function $\chi : G \to \{ z\in \mathbb{C} : |z| = 1 \} = \mathbb{K} $

is called a semigroup character if it is continuous and $\chi(g_1g_2) = \chi (g_1)\chi (g_2)$ for

all $g_1, g_2 \in G$. A semitopological semigroup $G$ is called right amenable

if the Banach lattice  $(C_b(G), \| \cdot \|_{\sup}) $ has a right invariant mean (i.e. there

exists on  a positive functional $m$ such that $\langle \mathbf{1} , m \rangle  = 1, $

and $\langle f, m \rangle = \langle f(\cdot g), m \rangle  $ for all $g\in G $ and all $f\in C_b(G)$

cf. [3], [7]).



Extending the notion of Cesaro averages (c.f. [2], [5], [6]) we say that

a net $(A_{\alpha }^{\mathcal{S}})_{\alpha }$ of contraction operators on $C(K)$

is called strong right $\mathcal{S}$-ergodic if

$A_{\alpha}^{\mathcal{S}} \in \overline{conv\mathcal{S}}^{s.o.t.}$ and

$\lim_{\alpha} \| A_{\alpha}^{\mathcal{S}}f -  A_{\alpha}^{\mathcal{S}}S_gf \|_{\sup} = 0$

for all $g\in G$ and $f\in C(K)$. The semigroup $\mathcal{S}$ is called mean ergodic if

$\overline{conv \mathcal{S}}^{s.o.t.}$ contains a (Markovian) zero element $Q$ (cf. [5], [8])

We denote $Fix(\mathcal{S}) = \{ f\in C(K) : S_gf = f  $ \ for all \ $g\in G \}$ and similarly

$Fix(\mathcal{S}') = \{ \nu\in C(K)' : S_g'\nu  = \nu  $ \ for all \ $g\in G \}$. If

for every $\nu \in Fix(\mathcal{S}')$ there exists $f\in Fix(\mathcal{S})$

such that $\langle f , \nu \rangle \neq 0 $ then we say that

$Fix(\mathcal{S})$ separates $Fix(\mathcal{S}')$.

Let us recall a characterization of strong mean ergodicity

for contraction (linear) semigroups (cf. [8], Theorem 1.7 and Corollary 1.8).

\begin{prop}

Let $G$ be represented on $C(K)$ by a right amenable semigroup of contractions

$\mathcal{S} = \{ S_g : g \in G \}$. Then the following conditions are equivalent:

\begin{itemize}

\item[(1)] $\mathcal{S}$ is mean ergodic with mean ergodic projection $P$,

\item[(2)] $Fix(\mathcal{S})$ separates $Fix(\mathcal{S}')$,

\item[(3)] $C(K) = Fix(\mathcal{S})\oplus \overline{rg(I - \mathcal{S})} $,

\item[(4)] $A_{\alpha}^{\mathcal{S}}f$ converges strongly (equivalently weakly)

to $Qf$ for some/every strong right $\mathcal{S}$-ergodic net

$A_{\alpha}^{\mathcal{S}}$ and all $f\in C(K)$.

\end{itemize}

\end{prop}



Given a semigroup character $\chi : G \to \mathbb{K} $ let $_{\chi}\mathcal{S}$

denote the semigroup $\{ \chi(g)S_g : g\in G\}$. The question whether mean ergodicity

of $\mathcal{S}$ is preserved when we pass to the distorted semigroup $_{\chi}\mathcal{S}$

was addressed in several papers (cf. [1], [9], [10] and [12]).



A Markovian semigroup $\mathcal{S}$ is called uniquely ergodic if

$dim(Fix(\mathcal{S}')) = 1$ (c.f. [2]). In this case there exists a unique probability

measure $\mu \in C(K)'$ such that $S_g'\mu = \mu $ for all $g\in G$. Clearly

unique ergodicity implies (cf. [11] Proposition 2.2) that $\mathcal{S}$ is mean

ergodic and $Fix(\mathcal{S}) = \mathbb{C}\mathbf{1}$. Even in this

situation, having a markovian representation $\mathcal{S}$ which is uniquely ergodic,

it may happen that for some characters $\chi $ the semigroup

$_{\chi}\mathcal{S}$  is not mean ergodic (cf. [9], [12]). The necessary and sufficient

condition guaranteeing mean ergodicity of $_{\chi}\mathcal{S}$ is formulated

in [10] in terms of yet another semigroup $_{\chi}\mathcal{S}_2$.

It is well known that the domain of any Markov operator $S$ may be  extended

by ($Sg(x) = \int g(y) S'\delta_x(dy)$)  to  all bounded and Borel measurable functions.

If $\mu $ is a $S'$ invariant probability, then this canonical extension appears to be a positive

linear contraction once acting on $L^2(\mu)$. Following [10] let $\mathcal{S}_2$ denote

the positive semigroup of linear contractions $S_g$ which are extended to $L^2(\mu )$.

Similarly $_{\chi}\mathcal{S}_2$ stands for all $\chi(g)S_g$, $g\in G$ which act

on $L^2(\mu )$. It has been recently proved in [10]



\begin{thm} (M. Schreiber) Let $\mathcal{S} = \{ S_g : g\in G \}$ be a

representation of a right amenable semigroup $G$ as Markov operators on

$C(K)$ and assume that $\mathcal{S}$ is uniquely ergodic with invariant

probability measure $\mu $. Then for a continuous character $\chi $ on $G$

the following conditions are equivalent:

\begin{itemize}

\item[(1)] $Fix(_{\chi}\mathcal{S}_2 ) \subseteq Fix(_{\chi}S)$,

\item[(2)] $_{\chi}\mathcal{S}$ is mean ergodic with mean ergodic projection $P_{\chi}$,

\item[(3)] $Fix(\mathcal{_{\chi}S})$ separates $Fix(_{\chi}\mathcal{S}')$,

\item[(3)] $C(K) = Fix(_{\chi}\mathcal{S})\oplus \overline{rg(I - \ _{\chi}\mathcal{S})} $,

\item[(4)] $A_{\alpha}^{_{\chi}\mathcal{S}}f$ converges strongly (equivalently weakly)

for some/every strong right $_{\chi}\mathcal{S}$-ergodic net $A_{\alpha}^{_{\chi}\mathcal{S}}$

and all $f\in C(K)$.

\end{itemize}

\end{thm}



\section{Result}



In this section we generalize the above result to mean ergodic Markov representations

without the unique ergodicity assumption. By $P(K)$ we denote  the

convex and *weak compact set of all probability (regular, Borel) measures on $K$.

We set $\mathbb{P}_{\mathcal{S}} = \{ \mu\in P(K) : S_g'\mu = \mu $ \ for all $g\in G \}$.

If $\mu \in \mathbb{P}_{\mathcal{S}} $ then  both $\mathcal{S}$ and $_{\chi}\mathcal{S}$

may be extended to $L^2(\mu )$. These extensions are

denoted respectively $\mathcal{S}_{2, \mu}$  or $_{\chi}\mathcal{S}_{2,\mu}$. Clearly

they all are contraction semigroups.   Now the version

of the Wiener-Wintner ergodic theorem may be formulated as follows:



\begin{thm}

Let $\mathcal{S} = \{ S_g : g\in G \}$ be a

representation of a right amenable semigroup $G$ as Markov operators on

$C(K)$. If $\mathcal{S}$ is mean ergodic then for any continuous character $\chi $ on $G$

the following conditions are equivalent:

\begin{itemize}

\item[(1)] $\overline{Fix(_{\chi}\mathcal{S})}^{L^2(\mu)} = Fix(_{\chi}\mathcal{S}_{2,\mu})$ for any

$\mu \in \mathbb{P}_{\mathcal{S}}$,

\item[(2)] $_{\chi}\mathcal{S}$ is mean ergodic with mean ergodic projection $Q_{\chi}$,

\item[(3)] $Fix(\mathcal{_{\chi}S})$ separates $Fix(_{\chi}\mathcal{S}')$,

\item[(4)] $C(K) = Fix(_{\chi}\mathcal{S})\oplus \overline{rg(I - \ _{\chi}\mathcal{S})} $,

\item[(5)] $A_{\alpha}^{_{\chi}\mathcal{S}}f$ converges strongly (equivalently weakly) to $_{\chi}Q$

for some/every strong right $_{\chi}\mathcal{S}$-ergodic net $A_{\alpha}^{_{\chi}\mathcal{S}}$,

and all $f\in C(K)$.

\end{itemize}

\end{thm}

Proof: It follows from the general abstract operator ergodic theorem (see Proposition 1.1)

that it is sufficient to prove equivalence of (1) and (3).



$(1) \Rightarrow (3)$ Let $\nu \in Fix(_{\chi}\mathcal{S}')$ be nonzero.

We have  $_{\chi}S_g'\nu = \nu $ or equivalently $S_g'\nu = \overline{\chi (g)}\nu$ for all

$g\in G$.

Since $S_g$ are positive linear contractions on the (complex) Banach lattice

$C(K)' = M(K)$ of regular finite (complex) measures on $K$ it follows that

$S_g'|\nu | \geq |S_g'\nu | = |\overline{\chi(g)}\nu | = |\nu |$, where

$|\cdot |$ denotes the lattice modulus in $M(K)$. Hence $S_g'|\nu | = |\nu |$ as $\| S_g' \| = 1$.

Without loss of generality we may assume that $|\nu | \in \mathbb{P}_{\mathcal{S}}$.

Clearly $\nu = \overline{g}|\nu |$ for some modulus 1 function $g$ and by Lemma 2.5 in [10]

$g\in Fix(_{\chi}\mathcal{S}_{2,|\nu |})$ (the assumption of unique ergodicity

is not required here). By (1) we find a sequence $g_n \in Fix(_{\chi}\mathcal{S})$ such that

$\| g_n - g \|_{L^2(| \nu |)} \to 0$. Now $\langle g_n , \nu \rangle  = \int g_n\overline{g}d|\nu |

\to \int g\overline{g}d|\nu | = 1$. Hence $ \langle g_n , \nu \rangle  \neq 0$ for some $n$.

It follows that  $Fix(\mathcal{_{\chi}S})$ separates $Fix(_{\chi}\mathcal{S}')$.



$(3) \Rightarrow (1)$ Suppose that there exists $\mu \in \mathbb{P}_{\mathcal{S}}$

such that $(1)$ fails. Then there exists $f\in Fix(_{\chi}\mathcal{S}_{2,\mu })$ such

that $f \perp Fix(_{\chi}\mathcal{S})$. Applying once again Lemma 2.5 from [10]

we have $\overline{f}\mu \in Fix(_{\chi}\mathcal{S}')$. By (3) there

exists $q \in Fix(_{\chi}\mathcal{S})$ such that $0 \neq \langle q, \overline{f}\mu \rangle

 = \int_K q\overline{f}d\mu = \langle q , f \rangle_{L^2(\mu)}$, a contradiction.



\bs



If $\mathcal{S}$ is uniquely ergodic then by Lemma 2.6 in [10]

$dim (Fix(_{\chi}\mathcal{S})) \leq 1 $ and therefore the closure operation

in condition $(1)$ is redundant. We end this note by extending Theorem 2.7 from [10]

(simultaneously simplifying its proof).

The general results on  unique ergodicity, strict ergodicity, irreducibility

and the structure of supports of invariant measures for

ergodic nets of Markov operators on $C(K)$ may be found in [2].



\begin{thm}

Let $\mathcal{S} = \{ S_g : g\in G \}$ be a

representation of a right amenable semigroup $G$ as Markov operators on

$C(K)$. If $\mathcal{S}$ is mean ergodic with finite dimensional

ergodic projection $Q$ then for any continuous character $\chi $ on $G$

the following conditions are equivalent:

\begin{itemize}

\item[(1)] $Fix(_{\chi}\mathcal{S}) = Fix(_{\chi}\mathcal{S}_{2,\mu})$ for all

$\mu \in \mathbb{P}_{\mathcal{S}}$,

\item[(2)] $_{\chi}\mathcal{S}$ is mean ergodic with mean ergodic projection $Q_{\chi}$,

\item[(3)] $Fix(\mathcal{_{\chi}S})$ separates $Fix(_{\chi}\mathcal{S}')$,

\item[(3)] $C(K) = Fix(_{\chi}\mathcal{S})\oplus \overline{rg(I - \  _{\chi}\mathcal{S})} $,

\item[(4)] $A_{\alpha}^{_{\chi}\mathcal{S}}f$ converges strongly (equivalently weakly) to

$Q_{\chi }$ for some/every strong right $_{\chi}\mathcal{S}$-ergodic net

$A_{\alpha}^{_{\chi}\mathcal{S}}$ and all $f\in C(K)$.

\end{itemize}

\end{thm}

Proof: Clearly condition $(1)$ here implies condition $(1)$ in Theorem 2.1.

Hence it is sufficient to prove:



(3) $ \Rightarrow$ (1)

Given a character $\chi $ on $G$ we shall  prove that

$dim Fix(_{\chi}\mathcal{S}) < \infty$. We assume that $\mathcal{S}$

is mean ergodic and that $dim Fix(\mathcal{S}) < \infty$. Hence there is

finitely many  extremal invariant probabilities  $\mu_1, \mu_2, \dots \mu_k

\in ex \mathbb{P}_{\mathcal{S}}$. It follows from the mean ergodicity of $\mathcal{S}$

that topological supports of $\mu_1, \dots \mu_k$ are disjoint (closed) subsets

of $K$. Let $C_{\mathcal{S}}$ be the union $\bigcup_{j=1}^k \ supp(\mu _j)$.

Clearly each set $supp(\mu_j)$ is $\mathcal{S}$-invariant (i.e.

$S'_g \delta_x (supp (\mu_j)) = 1$ for all $g\mathcal{\in G}$ and $x\in supp(\mu_j)$,

$j =1, ... , k$). Let $\mu = \frac{1}{k}(\mu_1 + ... + \mu_k) \in \mathbb{P}_{\mathcal{S}}$.

If $f\in Fix(_{\chi}\mathcal{S})\subseteq L^2(\mu )$ then $\chi(g)S_gf = f $ and

therefore $S_gf = \overline{\chi(g)}f$. Considering $S_g$ as a linear contraction on $L^2(\mu )$

we get $S_g|f| = |f| \ \mu \ a.e.$. In particular, $S_g|f| = |f|$ on  $C_{\mathcal{S}}$.

Hence on each support $supp(\mu_j)$ the function $|f|$ is constant.

Let us take arbitrary $x\in supp(\mu_j)$  and $g\in G$. We have

$$

f(x) =  \chi(g)\int_K f(y) S_g'\delta_x(dy) = \chi(g)\int_{supp(\mu_j)} f(y) S_g'\delta_x(dy) \ .

$$

Hence $f(y) = \overline{\chi(g)}f(x)$ for $y \in supp(S_g'\delta_x)$.

It follows that for  any $f_1, f_2 \in Fix(_{\chi}\mathcal{S})$ such that

$f_2 \neq 0$ on $supp(\mu_j)$ and all $x\in supp(\mu_j)$, $g\in G$ we have

$$

S_g\left(\frac{f_1}{f_2}\right) (x) = \int_{supp(\mu_j)} \frac{f_1(y)}{f_2(y)} S_g'\delta_x(dy) =

\frac{\overline{\chi(g)}f_1(x)}{\overline{\chi(g)}f_2(x)} = \frac{f_1}{f_2}(x).

$$

Since $\mathcal{S}$-invariant functions are constant on supports of extremal

invariant probabilities, thus $\frac{f_1}{f_2} = c$ on $supp(\mu_j)$. In other words if

$f_1, f_2 \in Fix(_{\chi}\mathcal{S})$ then $f_1 = \alpha_j(f_1, f_2)f_2$ on $supp(\mu_j)$ or simply

$dim Fix(_{\chi}\mathcal{S})\mathbf{1}_{supp \mu_j} = 1$ for any $j = 1, ... , k$.





Let $f_j\in Fix(_{\chi}\mathcal{S})\mathbf{1}_{supp \mu_j}$ be nonzero (as long as such a function

exists). Then any

$f\in Fix(_{\chi}\mathcal{S})$ may be represented  in  $L^2(\mu)$ as

$f = \sum_{j=1}^k f \mathbf{1}_{supp \mu_j} = \sum_{j=1}^k \alpha_jf_j\mathbf{1}_{supp \mu_j}$.

In particular, regardless of the choice of invariant $\mu \in Fix(\mathcal{S}')$ the estimation

$dim Fix(_{\chi}\mathcal{S}) \leq dim Fix(\mathcal{S}) = k $ in $L^2(\mu )$ holds true.

Hence using Theorem 2.1 the condition (3) implies

$$

Fix(_{\chi}\mathcal{S}_{2,\mu}) = \overline{Fix(_{\chi}\mathcal{S})}^{L^2(\mu)} = Fix(_{\chi}\mathcal{S}) \

(\mu \ a.e.)

$$

for all $\mu \in \mathbb{P}_{\mathcal{S}}$.



\bs



\begin{thebibliography}{12}

\bibitem {} \ I. Assani, {\it Wiener Wintner Ergodic Theorems}, World Scientific,

River Edge, NJ, 2003.

\bibitem {} \ W. Bartoszek and N. Erkursun, {\it On quasi-compact Markov nets,}

Erg. Th. \& Dynam. Sys. {\bf 31} (2011) 1081--1094.

\bibitem {} \ M.M. Day, {\it Amenable semigroups}, Illinois J. Math., vol. 1 (1957),

pp. 509--544.

\bibitem {} \ T.Eisner, B. Farkas, M. Haase and R. Nagel, {\it Operator Theoretic Aspects

of Ergodic Theory}, Springer (2014).

\bibitem {} \ E.Yu Emel'yanov and N. Erkursun, {\it Generalization of Eberlein's

and Sine's ergodic theorem to $(\mathcal{L}, \mathcal{R})$-nets}, Vladikavkaz Mat. Zh.

{\bf 9} (3) (2007), 22-26.

\bibitem {} \ E.Yu Emel'yanov and N. Erkursun, {\it Lotz-R\"abiger's nets of Markov

operators on $L^1-$spaces}, J. Math. Anal. Appl. {\bf 371} (2010) no. 2, 777-783.

\bibitem {} \ F. Greenleaf, {\it Invariant Means on Topological Groups

and their Applications}, Van Nostrand New York (1969).

\bibitem {} \ U. Krengel, {\it Ergodic Theorems (de Gruyter Studies

in Mathematics, 6)}, Walter de Gruyter \& Co. Berlin, (1985).

201--217.

\bibitem {} \ E.A. Robinson, {\it On uniform convergence in the Wiener-Wintner theorem},

J. Lond. Math. Soc. {\bf 49} (1994), 493--501.

\bibitem {}  \ M. Schreiber, {\it Topological Wiener-Wintner theorems for

amenable operator semigroups}, Ergod. Th. \& Dynam. Sys. {\bf 34}

(2014) no. 5, 1674--1698.

\bibitem {} \ M. Schreiber, {\it Uniform familes of ergodic operator nets}, Semigroup Forum

{\bf 86} (2013) no. 2, 321--336.

\bibitem {} \ P. Walters, {\it Topological Wiener-Wintner ergodic theorems

and a random $L^2$ ergodic theorem}, Ergod. Th. \& Dynam. Sys. {\bf 16} (1996) 179--206.



\end{thebibliography}

\end{document}

