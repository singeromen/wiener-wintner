%
% Niniejszy plik stanowi przyk�ad formatowania pracy magisterskiej na
% Wydziale MIM UW.  Szkielet u�ytych polece� mo�na wykorzystywa� do
% woli, np. formatujac wlasna prace.
%
% Zawartosc merytoryczna stanowi oryginalnosiagniecie
% naukowosciowe Marcina Wolinskiego.  Wszelkie prawa zastrze�one.
%
% Copyright (c) 2001 by Marcin Woli�ski <M.Wolinski@gust.org.pl>
% Poprawki spowodowane zmianami przepis�w - Marcin Szczuka, 1.10.2004
% Poprawki spowodowane zmianami przepisow i ujednolicenie 
% - Seweryn Kar�owicz, 05.05.2006
% dodaj opcj� [licencjacka] dla pracy licencjackiej
\documentclass{pracamgr}
\usepackage{amsmath}
\usepackage{amsfonts}
\usepackage[a4paper, left=2.5cm, right=2.5cm, top=3.5cm, bottom=3.5cm, headsep=1.2cm]{geometry}
\usepackage{graphicx}
\usepackage{float}
\usepackage{enumerate}
\usepackage{ mathrsfs }
\usepackage{ amssymb }
\usepackage{amsthm}
\usepackage{etoolbox}
\usepackage{ dsfont }
\usepackage{ hyperref }
\usepackage{mathtools}

%Jesli uzywasz kodowania polskich znakow ISO-8859-2 nastepna linia powinna byc 
%odkomentowana
%\usepackage[latin2]{inputenc}
%Jesli uzywasz kodowania polskich znakow CP-1250 to ta linia powinna byc 
%odkomentowana
\usepackage[cp1250]{inputenc}

% Dane magistranta:

\author{Adam �piewak}

\nralbumu{132528}

\title{Generalizations of Wiener-Wintner ergodic theorem}

\tytulang{Rozszerzenia twierdzenia ergodycznego Wienera-Wintnera}

%kierunek: Matematyka, Informatyka, ...
\kierunek{Matematyka}

% informatyka - nie okreslamy zakresu (opcja zakomentowana)
% matematyka - zakres moze pozostac nieokreslony,
% a jesli ma byc okreslony dla pracy mgr,
% to przyjmuje jedna z wartosci:
% {metod matematycznych w finansach}
% {metod matematycznych w ubezpieczeniach}
% {matematyki stosowanej}
% {nauczania matematyki}
% Dla pracy licencjackiej mamy natomiast
% mozliwosc wpisania takiej wartosci zakresu:
% {Jednoczesnych Studiow Ekonomiczno--Matematycznych}

% \zakres{Tu wpisac, jesli trzeba, jedna z opcji podanych wyzej}

% Praca wykonana pod kierunkiem:
% (poda� tytu�/stopie� imi� i nazwisko opiekuna
% Instytut
% ew. Wydzia� ew. Uczelnia (je�eli nie MIM UW))
\opiekun{prof. dra hab. in�. Wojciecha Bartoszka\\
  Katedra Rachunku Prawdopodobie�stwa i Biomatematyki\\
  }

% miesi�c i~rok:
\date{Maj 2015}

%Poda� dziedzin� wg klasyfikacji Socrates-Erasmus:
\dziedzina{ 
%11.0 Matematyka, Informatyka:\\ 
11.1 Matematyka\\ 
%11.2 Statystyka\\ 
%11.3 Informatyka\\ 
%11.4 Sztuczna inteligencja\\ 
%11.5 Nauki aktuarialne\\
%11.9 Inne nauki matematyczne i informatyczne
}

%Klasyfikacja tematyczna wedlug AMS (matematyka) lub ACM (informatyka)
\klasyfikacja{37 Dynamical systems and ergodic theory\\
  37A Ergodic theory\\
  37A30 Ergodic theorems, spectral theory, Markov operators}

% S�owa kluczowe:
\keywords{teoria ergodyczna}

% Tu jest dobre miejsce na Twoje w�asne makra i~�rodowiska:
\newtheoremstyle{dotless}{}{}{}{}{\bfseries}{}{ }{}
\newtheoremstyle{withspace}{}{}{}{}{\bfseries}{ }{ }{}
\newtheoremstyle{colon}{0}{}{}{}{\bfseries}{:}{ }{}

\newtheoremstyle{mytheorem}%                % Name
  {}%                                     % Space above
  {}%                                     % Space below
  {\itshape}%                             % Body font
  {}%                                     % Indent amount
  {\bfseries}%                            % Theorem head font
  {}%                                    % Punctuation after theorem head
  {\newline}%                                    % Space after theorem head, ' ', or \newline
  {}%                                     % Theorem head spec (can be left empty, meaning `normal')

\theoremstyle{mytheorem}
\newtheorem{thm}{Theorem}[chapter]
\newtheorem{lem}{Lemma}[chapter]
\newtheorem{col}{Corollary}[chapter]

\theoremstyle{withspace}
\newtheorem{df}{Definition}[chapter]
\newtheorem{prop}{Proposition}[chapter]
\newtheorem{ex}{Example}[chapter]

\theoremstyle{dotless}
\newtheorem*{rem}{Remark}


\theoremstyle{colon}
\newtheorem*{pf}{Proof}

\AfterEndEnvironment{thm}{\noindent\ignorespaces}
\AfterEndEnvironment{lem}{\noindent\ignorespaces}
\AfterEndEnvironment{df}{\noindent\ignorespaces}
\AfterEndEnvironment{prop}{\noindent\ignorespaces}
\AfterEndEnvironment{ex}{\noindent\ignorespaces}
\AfterEndEnvironment{rem}{\noindent\ignorespaces}
\AfterEndEnvironment{pf}{\noindent\ignorespaces}

% koniec definicji

\begin{document}
\maketitle

%tu idzie streszczenie na strone poczatkowa
\begin{abstract}
  W pracy przedstawiono klasyczne twierdzenie ergodyczne Wienera-Wintnera wraz z licznymi rozszerzeniami.
\end{abstract}

\tableofcontents
%\listoffigures
%\listoftables

\chapter*{Introduction}
\addcontentsline{toc}{chapter}{Introduction}
Twierdzenie ergodycznie Wienera-Wintnera jest bardzo wa�ne. Bardzo bardzo wa�ne.
%%% Local Variables:
%%% mode: latex
%%% TeX-master: t
%%% coding: latin-2
%%% End:






\chapter{Preliminaries}\label{r:prel}
In this chapter we introduce basic notations, concepts and theorems from measure theory, topology and functional analysis which will be used through the thesis. We omit most of the proofs.\\ \\
By $\mathbb{N}$ we will denote set of positive natural numbers, by $\mathbb{N}_0$ - set of natural numbers together with zero, by $\mathbb{Z}$ - set of integers, by $\mathbb{R}$ - set of real numbers, by $\mathbb{C}$ - set of complex numbers and by $\mathbb{T}=\mathbb{S}^1=\left\{ \lambda \in \mathbb{C}: |\lambda|=1 \right\}$ - circle on a complex plane (1-dimensional torus).



\section{Measure theory}

complex measure and integral?!\\
finite and $\sigma$-finite measure spaces\\
miara Lebesgue'a m\\
miary produktowe\\
twierdzenie fubiniego\\
a.e. convergence\\
lebesgue's dominated convergence theorem\\
spaces $\mathcal{L}^1(\mu)$ and $L^1(\mu)$\\
absolute continuity and Radon-Nikodym theorem\\

\begin{df}
Let $X$ be nonempty set. Family of sets $\mathcal{A} \subset 2^X$ is called $\sigma$\textbf{-field} (or $\sigma$\textbf{-algebra}), when the following conditions hold:
\begin{enumerate}[(i)]
\item $\emptyset \in \mathcal{A},$
\item $A \in \mathcal{A} \Longrightarrow A' \in \mathcal{A}$,
\item $A_n \in \mathcal{A} \text{ for } n \in \mathbb{N}_0 \Longrightarrow \bigcup \limits_{n=0}^{\infty} A_n \in \mathcal{A}$.
\end{enumerate}
Pair $(X, \mathcal{A})$ is called a \textbf{measurable space}. Set $A \in \mathcal{A}$ is called a \textbf{measurable set}.
\end{df}

\begin{df}
Let $(X,\mathcal{A})$ be a measurable space. Function $\mu : \mathcal{A} \rightarrow [0, +\infty]$ is called a \textbf{(non-negative) measure} if it satifises the following properties:
\begin{enumerate}[(i)]
\item $\mu(\emptyset) = 0$,
\item $\mu\left( \bigcup \limits_{n=0}^{\infty} A_n \right) = \sum \limits_{n=0}^{\infty} \mu (A_n)$ for any countable collection of measurable sets $A_n \in \mathcal{A},\ n \in \mathbb{N}_0$.
\end{enumerate}
Triple $(X, \mathcal{A}, \mu)$ is called a \textbf{measure space}.
\end{df}

\begin{df}
Let $(X, \mathcal{A})$ and $(Y, \mathcal{C})$ be measurable spaces. A map $T: X \rightarrow Y$ is called a \textbf{measurable map} if it satisfies $T^{-1}(C) \in \mathcal{A} \text{ for all } C \in \mathcal{C}$.
\end{df}

\begin{df}
Let $(X, \mathcal{A}, \mu)$ be a measure space. An element $x \in X$ is called an \textbf{atom} (of the measure $\mu$) if $\mu(\{ x\})>0$. The measure $\mu$ is called \textbf{continuous} if it has no atoms, i.e. $\underset{x \in X}{\forall}\ \mu(\{ x\})=0$.
\end{df}

\begin{rem} Note that a finite measure $\mu$ can have only countably many atoms. To see that observe that for $\varepsilon >0$ a set $A_{\varepsilon}:=\{x \in X: \mu(\{x\}) > \epsilon \}$ must have at most $\frac{\mu(X)}{\varepsilon}$ elements (otherwise we wolud have $\mu(X) > \frac{\mu(X)}{\varepsilon} \cdot \varepsilon = \mu(X)$), hence must be finite. This shows that the set of atoms $A = \bigcup \limits_{n=1}^{\infty} A_{\frac{1}{n}}$ must be countable. Also, there is $\sum \limits_{x \in A} \mu(\{ x\}) \leq \mu(X) < \infty.$
\end{rem}


\section{Topology}
topological space\\
metric space\\
continuous map\\
compact space, complete metric space\\
Urysohn lemma\\
Borel measures\\



\section{Functional analysis}
convergence of geometric series on crircle\\
dual space\\
Riesz theorem (Hilbert spaces)\\
Banach and Hilbert conjugate\\
Riesz-Markov theorem\\

In the following we will always assume that vector spaces are taken over field $\mathbb{C}$.

\begin{df}
Let $E$ be a vector space. We say that a function $\| \cdot \| : E \rightarrow [0, \infty)$ is a \textbf{norm}, if for all $x,y \in E$ the following conditions are satisfied:
\begin{enumerate}[(i)]
\item $\|x\| = 0 \Leftrightarrow x=0$,
\item $\|\lambda x \| = |\lambda| \| x \|$ for all $\lambda \in \mathbb{C}$,
\item $\|x + y\| \leq \|x\| + \|y\|$.
\end{enumerate}
Vector space $E$ equipped with a norm is called a \textbf{normed space}.
\end{df}

\begin{rem} Note that a norm gives rise to a metric on $E$ by $d(x,y) = \| x - y\|$. This metric generates a topology on $E$ which is considered as a standard topology on $E$. Convergence in this metric is called convergence in norm (or strong convergence) and is sometimes denoted by $x_n \stackrel{\| \cdot \|}{\longrightarrow} x$ or $x_n \stackrel{E}{\longrightarrow} x$.
\end{rem}

\begin{df}
Let $E$ be a normed space. If $E$ is complete as a metric space, then $E$ is called a \textbf{Banach space}.
\end{df}

\begin{ex}[$\mathscr{L}^p$ and $L^p$ spaces]$\\$
Let $(X, \mathcal{A}, \mu)$ be a measure space. For $1\leq p < \infty$ consider the vector space \[\mathscr{L}^p(X, \mathcal{A}, \mu) := \left\{ f:X \rightarrow \mathbb{C};\ f \text{ is measurable and } \int \limits_{X} |f|^p d\mu < \infty \right\}.\]
Define an equivalence relation $\sim$ on $\mathscr{L}^p(X, \mathcal{A}, \mu)$ by $f \sim g$ if $f = g\ \mu$ a.e. Let
\[ L^p(X, \mathcal{A}, \mu) := \mathscr{L}^p(X, \mathcal{A}, \mu) / \sim. \]
Spaces $L^p(X, \mathcal{A}, \mu)$ are considered with norm $\|f\|_{L^p(X, \mathcal{A}, \mu)} :=  \left( \int \limits_{X} |f|^p d\mu \right) ^ {\frac{1}{p}}$ with which they become Banach spaces. Usually we will abbreviate $L^p(X, \mathcal{A}, \mu)$ to $L^p(\mu)$ or $L^p$ and $\|\cdot\|_{L^p(X, \mathcal{A}, \mu)}$ to $\| \cdot \|_{L^p(\mu)}$ or $\|\cdot\|_p$.
\end{ex}

\begin{ex}[Space $C(X)$]\label{ex:lp}$\\$
Let $X$ be a compact metric space. Denote by $C(X)$ set of all complex valued continuous functions on $X$. $C(X)$ is a Banach space with norm $\|f\|_{\text{sup}} = \|f\|_{\infty} := \sup \limits_{x \in X} |f(x)|,\ f \in C(X)$. Suppose that there is a finite Borel nonegative measure $\mu$ on $X$. Any function $f \in C(X)$ is bounded, hence integrable with any power $p \in [1, \infty)$, which means that $C(X) \subset \mathscr{L}^p(\mu)$ and $C(X)$ can be embedded into $L^p(\mu)$. Therefore, space $C(X)$ can be naturally seen as a linear subspace of space $L^p(\mu)$ (with identification of functions equal $\mu$ a.e.).
\end{ex}

\begin{prop}
Let $X$ be a compact metric space and $\mu$ be a finite nonegative Borel measure on $X$. Then $C(X)$ is dense in $L^p(\mu)$ (in $L^p(\mu)$ norm) for every $p \in [1, \infty)$.
\end{prop}
\begin{pf} CZY DOW�D?
\end{pf}


\begin{df}
Let $H$ be a vector space. A function $\langle \cdot,\cdot \rangle:H \times H \rightarrow \mathbb{C}$ is called a \textbf{inner product} if for all $x,y,z \in H$ the following conditions are satisified:
\begin{enumerate}[(i)]
\item $\langle x,x\rangle > 0 \text{ for }x \neq 0$,
\item $\langle x+y, z \rangle = \langle x, z \rangle + \langle y, z \rangle$,
\item $\langle \lambda x, y \rangle = \lambda \langle x, y \rangle \text{ for all } \lambda \in \mathbb{C}$,
\item $\langle x, y \rangle = \overline{\langle y, x \rangle}$.
\end{enumerate}
Vector space $H$ with inner product is called \textbf{inner product space}.
\end{df}

\begin{rem} Inner product space is a normed space with a norm $\|x\| := \sqrt{\langle x, x \rangle}$.
\end{rem}

\begin{df}Inner product space $H$ which is a Banach space is called a \textbf{Hilbert space}.
\end{df}

\begin{ex}[Space $L^2(\mu)$]$\\$
Let $(X, \mathcal{A}, \mu)$ be a measure space. The space $L^2(\mu)$ with inner product $\langle f,g\rangle := \int \limits_{X} f \overline{g} d\mu$ is a Hilbert space. Note that the inner product norm coincides with norm $\| \cdot \|_{L^2(\mu)}$ from Example \ref{ex:lp}.
\end{ex}

\begin{prop}[Cauchy�Schwarz inequality]$\\$
Let $H$ be an inner product space. The following inequality holds for all $x,y \in H$:
\[ |\langle x, y \rangle | \leq \|x\| \|y\|. \]
\end{prop}

\begin{rem} Cauchy-Schwarz inequality implies that inner product is a continous map in each variable.
\end{rem}

\begin{df}
Let $H$ be an inner product space. Two vectors $x, y \in H$ are said to be \textbf{orthogonal} if $\langle x, y \rangle = 0$. We denote that fact by $x \bot y$. For a set $H_0 \subset H$ its \textbf{orthogonal complement} is a set $H_0^{\bot}:=\left\{ x \in H: \underset{h \in H_0}{\forall}\ \langle h, x \rangle = 0 \right\}$.
\end{df}

\begin{rem} If $H_0$ is a linear subspace of H, then $H_0^{\bot}$ is a closed linear subspace of H. Closedness of $H_0^{\bot}$ is a consequence of continuity of the inner product.
\end{rem}

\begin{df}
Let $E, F$ be normed spaces. A linear transformation $U: E \rightarrow F$ is called a \textbf{bounded linear operator} if there exists $M > 0$ such that $\underset{x \in E}{\forall}\ \|Ux\| \leq M\|x\|$. Constant $\|U\| := \sup \limits_{\|x\| \leq 1} \|Ux\|$ is called a \textbf{operator norm} of U. If $\| U \| \leq 1$ then $U$ is called a \textbf{contraction}. If $\underset{x \in E}{\forall}\ \|Ux\| = \|x\|$ then $U$ is called an \textbf{isometry}. Note that an isometry is always a contraction.
\end{df}

\begin{rem} Linear operator $U: E \rightarrow F$ between normed spaces is continuous if and only if it's bounded. Space $L(E,F)$ of all bounded linear operators with the operator norm is a normed space. $L(E,F)$ is a Banach space if and only if $F$ is a Banach space.
\end{rem}

\begin{df}
Let $E$ be a normed space and let $U:E \rightarrow E$ be a bounded linear operator. Number $\lambda \in \mathbb{C}$ is called an \textbf{eigenvalue} if there is a vector $x \in E,\ x \neq 0$ such that $Ux = \lambda x$. Any such vector $x$ is called an \textbf{eigenvector} (associated with $\lambda$). The closed linear subspace $H_{\lambda} = \left\{ x \in H: Ux = \lambda x \right\}$ is called an \textbf{eigenspace} (of $\lambda$).
\end{df}

\begin{thm}[Orthogonal Projection Theorem �R�D�O!]
Let $H_0$ be a closed linear subspace of a Hilbert space $H$. Then
\[ H = H_0 \oplus H_0^{\bot}, \]
i.e. for every $x \in H$ there are unique $x_0 \in H_0,\ x_1 \in H_0^{\bot}$ such that $x = x_0 + x_1$. Moreover, transformation $P: H \rightarrow H$ given by $P(x)=x_0$ is a bounded linear operator with $\| P \| \leq 1$ and $P \circ P =P$. Operator $P$ is called an \textbf{orthogonal projection} on subspace $H_0$.
\end{thm}
We will now introduce basic facts from spectral theory for isometries on Hilbert spaces.

\begin{rem}Let $H$ be a complex inner product space. Then bounded linear operator $U:H \rightarrow H$ is an isometry if and only if $\underset{x,y \in H}{\forall}\ \langle Ux, Uy\rangle = \langle x, y \rangle$.
\end{rem} 

\begin{df}
Sequence $(r_n)_{n \in \mathbb{Z}}$ of complex numbers is called \textbf{positive definite} if for every sequence $(a_n)_{n \in \mathbb{N}_0}$ of complex numbers and every $N \in \mathbb{N}_0$ we have $\sum \limits_{n,m=0}^N r_{n-m}a_n \overline{a_m} \geq 0$.
\end{df}

\begin{prop} \label{p:posdef}
Let $U:H \rightarrow H$ be an isometry on Hilbert space $H$. For a vector $x \in H$ define $r_n := \langle U^n x, x \rangle$ for $n \geq 0$ and $r_n := \overline{r_{-n}} = \langle x, U^n x \rangle$ for $n<0$. The sequence $(r_n)_{n \in \mathbb{Z}}$ is positive definite.
\end{prop}
\begin{pf} Note first that for $n \geq m$ we have $r_{n-m} = \langle U^{n-m}x, x \rangle = \langle U^n x, U^m x \rangle$ (since $U$ is an isometry) and for $n < m$ we also have $r_{n-m} = \overline{r_{m-n}} = \overline{\langle U^{m-n} x, x \rangle} = \overline{\langle U^m x, U^n x \rangle} = \langle U^n x, U^m x \rangle$. Compute now
\begin{equation} \label{e:posdef}
\begin{split}
\sum \limits_{n,m=0}^N r_{n-m}a_n \overline{a_m} =  \sum \limits_{n,m=0}^N \langle U^n x, U^m x \rangle a_n \overline{a_m} = \sum \limits_{n,m=0}^N \langle a_n U^n x, a_m U^m x \rangle = \\
= \sum \limits_{n=0}^N \langle a_n U^n x, \sum \limits_{m=0}^N a_m U^m x \rangle = \langle \sum \limits_{n=0}^N  a_n U^n x, \sum \limits_{m=0}^N a_m U^m x \rangle = \| \sum \limits_{n=0}^N  a_n U^n x \|^2 \geq 0.
\end{split}
\end{equation}
\begin{flushright}
$\square$
\end{flushright}
\end{pf}

\begin{thm}[Herglotz's theorem {\cite[thm. 2.3]{lemanczyk}}]
Let $(r_n)_{n \in \mathbb{Z}}$ be positive definite sequence. There exists unique non-negative finite Borel measure $\sigma$ on $\mathbb{T}$ such that
\begin{equation} \label{e:spectral}
r_n = \int \limits_{\mathbb{T}} z^n d\sigma (z)\ \text{ for all } n\in \mathbb{Z}.
\end{equation}
Conversly, for every non-negative finite Borel measure $\sigma$ on $\mathbb{T}$, sequence $r_n$ defined by \eqref{e:spectral} is positive definite.
\end{thm}

\begin{df} Let $\sigma$ be a non-negiative finite Borel measure on $\mathbb{T}$. Then the number
\[ \hat{\sigma}(n):=  \int \limits_{\mathbb{T}} z^n d\sigma (z),\ n \in \mathbb{Z}\]
is called the \textbf{n-th Fourier coefficient} of the measure $\sigma$. Note that the sequence $\hat{\sigma}(n),\ n \in \mathbb{Z}$ is positive definite and $\hat{\sigma}(-n) = \overline{\hat{\sigma}(n)}$ for every $n \in \mathbb{Z}$.
\end{df}

\begin{col}[Spectral measure]
Let $U:H \rightarrow H$ be an isometry on Hilbert space $H$. For every vector $x \in H$ there exists unique non-negative finite Borel measure $\sigma_x$ on $\mathbb{T}$ such that
\[ \langle U^n x, x \rangle = \int \limits_{\mathbb{T}} z^n d\sigma_x (z)\ \text{  and  }\ \langle x, U^n x \rangle = \int \limits_{\mathbb{T}} z^{-n} d\sigma_x (z)\ \text{  for all } n\in \mathbb{N}_0. \]
The measure $\sigma_x$ is called a \textbf{spectral measure} of an element $x$.
\end{col}

\begin{prop}\label{p:spectreq}
Let $U:H \rightarrow H$ be an isometry on Hilbert space $H$. For every $x \in H$ and finite sequence $(a_n)_{n=0}^N$ of complex numbers the following equality holds:
\[ \| \sum \limits_{n=0}^N a_n U^n x \|^2 = \int \limits_{\mathbb{T}} | \sum \limits_{n=0}^N a_n  z^n |^2 d\sigma_x(z) = \|\sum \limits_{n=0}^N a_n z^n \|_{L^2(\mathbb{T}, \mathcal{B}(\mathbb{T}), \sigma_x)}^2. \]
\end{prop}
\begin{pf} For sequence $(r_n)_{n \in \mathbb{Z}}$ like in Proposition \ref{p:posdef}, we have by equalities \eqref{e:posdef} and \eqref{e:spectral}
\[ \|\sum \limits_{n=0}^N a_n U^n x \|^2 = \sum \limits_{n,m=0}^N r_{n-m}a_n \overline{a_m} = \sum \limits_{n,m=0}^N  a_n \overline{a_m}  \int \limits_{\mathbb{T}} z^{n-m} d\sigma_x (z) = \sum \limits_{n,m=0}^N  a_n \overline{a_m}  \int \limits_{\mathbb{T}} z^n\overline{z^m} d\sigma_x (z) =\]
\[ = \sum \limits_{n=0}^N a_n \int \limits_{\mathbb{T}} z^n (\sum \limits_{m=0}^N \overline{a_m z^m}) d\sigma_x(z) = \int \limits_{\mathbb{T}} \sum \limits_{n=0}^N a_n  z^n (\sum \limits_{m=0}^N \overline{a_m z^m}) d\sigma_x(z) = \int \limits_{\mathbb{T}} | \sum \limits_{n=0}^N a_n  z^n |^2 d\sigma_x(z).\]
\begin{flushright}
$\square$
\end{flushright}
\end{pf}
In order to prove Wiener's Criterion of Continuity, we need the following lemma (also due to Wiener):

\begin{lem}[Wiener, {\cite[lemma 1.16]{lemanczyk}}]
Let $\sigma$ be a finite non-negative Borel measure on $\mathbb{T}$. Denote by $\{a_1, a_2, ... \}$ a set of all atoms of measure $\sigma$. Then
\[ \lim \limits_{N \rightarrow \infty} \frac{1}{N} \sum \limits_{n=0}^{N-1} | \hat{\sigma} (n) |^2 = \lim \limits_{N \rightarrow \infty} \frac{1}{N} \sum \limits_{n=0}^{N-1} | \hat{\sigma} (-n) |^2 = \sum \limits_{m \geq 1} \sigma(\{ a_m \})^2. \]
\end{lem}
\begin{pf} Note first, that since $\hat{\sigma} (n) = \overline{\hat{\sigma}(-n)}$, limits $\lim \limits_{N \rightarrow \infty} \frac{1}{N} \sum \limits_{n=0}^{N-1} | \hat{\sigma} (n) |^2$ and $\lim \limits_{N \rightarrow \infty} \frac{1}{N} \sum \limits_{n=0}^{N-1} | \hat{\sigma} (-n) |^2$ must be equal if they exists. Note further, that since measure $\sigma$ is finite, series $\sum \limits_{m \geq 1} \sigma(\{ a_m \})^2$ must be convergent (we know that $\sum \limits_{m \geq 1} \sigma(\{ a_m \}) < \infty$ and only for finitely many $m \in \mathbb{N}$ there can be $\sigma(\{ a_m \}) \geq 1$). Observe that by Fubini's Theorem we have
\[ |\hat{\sigma} (n) |^2 = \hat{\sigma}(n) \overline{\hat{\sigma}(n)} = \int \limits_{\mathbb{T}} z^n d\sigma (z) \overline{ \int \limits_{\mathbb{T}} w^n d\sigma (w)} = \int \limits_{\mathbb{T}} z^n \left( \int \limits_{\mathbb{T}} \overline{w}^n d\sigma (w) \right) d\sigma (z) = \]
\[= \int \limits_{\mathbb{T} \times \mathbb{T}} (z \overline{w})^n d\sigma \otimes \sigma (z, w),\]
and further
\begin{equation}\label{e:fourier}
\frac{1}{N} \sum \limits_{n=0}^{N-1} | \hat{\sigma} (n) |^2 = \int \limits_{\mathbb{T}^2} \frac{1}{N} \sum \limits_{n=0}^{N-1} (z \overline{w})^n d\sigma \otimes \sigma (z, w).
\end{equation}
For $z, w \in \mathbb{T}$ we have also $z\overline{w} \in \mathbb{T}$ and $\lim \limits_{N \rightarrow \infty} \frac{1}{N} \sum \limits_{n=0}^{N-1} (z \overline{w})^n = \mathds{1}_{\{ (z,w) \in \mathbb{T}^2 :\ z\overline{w} = 1\}} (z, w) = \mathds{1}_{\Delta} (z,w),$ where $\Delta = \{ (z,w) \in \mathbb{T}^2:\ z = w \}$. Since $ | \frac{1}{N} \sum \limits_{n=0}^{N-1} (z \overline{w})^n |\leq \frac{1}{N} \sum \limits_{n=0}^{N-1} | (z \overline{w})^n| = 1$, we have by Lebesgue Dominated Convergence Theorem
\begin{equation}\label{e:delta}
\lim \limits_{N \rightarrow \infty} \int \limits_{\mathbb{T}^2} \frac{1}{N} \sum \limits_{n=0}^{N-1} (z \overline{w})^n d\sigma \otimes \sigma (z, w) = \int \limits_{\mathbb{T}^2} \mathds{1}_{\Delta} (z,w) d\sigma \otimes \sigma (z,w).
\end{equation}
By Fubini's Theorem we have
\[ \int \limits_{\mathbb{T}^2} \mathds{1}_{\Delta} (z,w) d\sigma \otimes \sigma (z,w) = \int \limits_{\mathbb{T}} \left( \int \limits_{\mathbb{T}} \mathds{1}_{\Delta} (z,w) d\sigma(w) \right) d\sigma(z)
= \int \limits_{\mathbb{T}} \left( \int \limits_{\mathbb{T}} \mathds{1}_{\{z\}} (w) d\sigma(w) \right) d\sigma(z) = \]
\[ = \int \limits_{\mathbb{T}} \sigma(\{ z\}) d\sigma (z) = \int \limits_{\bigcup \limits_{m \geq 1} \{ a_m \}} \sigma(\{ z\}) d\sigma (z) = \sum \limits_{m \geq 1} \sigma(\{ a_m \})^2, \]
what combined with \eqref{e:fourier} and \eqref{e:delta} completes the proof. $\square$
\end{pf}
\begin{col}[Wiener's Criterion of Continuity]\label{c:wiener}
Non-negative finite Borel measure $\sigma$ on $\mathbb{T}$ is continuous if and only if $\lim \limits_{N \rightarrow \infty} \frac{1}{N} \sum \limits_{n=0}^{N-1} | \hat{\sigma} (n) |^2 =\lim \limits_{N \rightarrow \infty} \frac{1}{N} \sum \limits_{n=0}^{N-1} | \hat{\sigma} (-n) |^2 = 0.\ \square$
\end{col}

After establishing von Neumann's Ergodic Theorem in next chapter, we will be able to prove another important lemma about spectral measures.







\chapter{Introduction to ergodic theory}\label{r:erg}

\section{Measurable dynamical systems}
measure preserving system\\
ergodic system and equivalences (at least invariant functions are constant)\\
Birkhoff theorem (for measure preserving systems) and note about using $L^1$ and $\mathscr{L}^1$\\
system $(\mathbb{T}, \mathcal{B}(\mathbb{T}), m, R_{\lambda})$\\
product systems\\
product of m.p.s. systems is m.p.s\\
Koopman operator - isometry, properties of eigenvalues and eigenfunctions\\



\section{Topological dynamical systems}




\section{von Neumann's Ergodic Theorem}
In this section we state and prove von Neumann's (Mean) Ergodic Theorem, which can be seen as a first operator theoretic type ergodic theorem.

\begin{thm}[von Neumann's Ergodic Theorem {\cite[thm. 1.3.1]{weber}}]\label{t:neumann}
Let $U:H \rightarrow H$ be a contraction on a complex Hilbert space $H$. Then for every $f \in H$ there is a convergence
\[ \lim \limits_{N \rightarrow \infty} \frac{1}{N} \sum \limits_{n=0}^{N-1} U^n f = Pf, \]
where $P:H \rightarrow H$ is an orthogonal projection to a closed subspace of $U$-invariant vectors $H_U = \left\{ g\in H: Ug = g \right\}$. Moreover, there is
\[ H=H_U \oplus H_0, \]
where $H_0 = \overline{\left\{ g - Ug: g \in H \right\}}$.
\end{thm}
\begin{pf} DOW�D!
\end{pf}
Note that this proof doesn't require use of spectral theory, although there is a simpler proof for unitary $U$ using spectral theorem (ODWO�ANIE DO RUDINA!). In the following lemma we will inverse this relationship and make use of von Neumann's theorem in spectral theory.

\begin{lem}\label{l:spectr}
Let $U:H\rightarrow H$ be an isometry on Hilbert space $H$ and take $f \in H$. Then $\sigma_f (\{\lambda\}) = \|P_{\lambda}f\|^2,$ where $\sigma_f$ denotes spectral measure of $f$ and $P_{\lambda}$ is an orthogonal projection to the $H_{\lambda}$ - the eigenspace of $\lambda \in \mathbb{T}$.
\end{lem}
\begin{pf} Note that operator is $V:H \rightarrow H$ given by $V:=\overline{\lambda}U$ is also an isometry, since $\langle Vf, Vg \rangle = \langle \overline{\lambda}Uf, \overline{\lambda}Ug \rangle= \overline{\lambda} \lambda \langle Uf, Ug \rangle = |\lambda|^2 \langle f, g \rangle = \langle f, g \rangle$. By von Neumann's Theorem we have that \[\frac{1}{N} \sum \limits_{n=0}^{N-1} \overline{\lambda}^n U^n f = \frac{1}{N} \sum \limits_{n=0}^{N-1} V^n f \longrightarrow Qf,\] where $Q$ is an orthogonal projection on a subspace $\left\{ f \in H: Vf = f \right\} = \left\{ f \in H: \overline{\lambda}Uf = f \right\} = \left\{ f \in H: Uf = \lambda f \right\} = H_{\lambda}$, so $Q = P_{\lambda}$. No we have 
\begin{equation}\label{e:norm}
\|\frac{1}{N} \sum \limits_{n=0}^{N-1} \overline{\lambda}^n U^n f \|^2 \rightarrow \|P_{\lambda} f \|^2,
\end{equation}
but from Proposition \ref{p:spectreq} we have also 
\begin{equation} \label{e:int}
\|\frac{1}{N} \sum \limits_{n=0}^{N-1} \overline{\lambda}^n U^n f \|^2 = \int \limits_{\mathbb{T}}  | \frac{1}{N} \sum \limits_{n=0}^{N-1} \overline{\lambda}^n z^n |^2 d\sigma_f (z) = \int \limits_{\mathbb{T}}  | \frac{1}{N} \sum \limits_{n=0}^{N-1}  \left(\frac{z}{\lambda}\right)^n |^2 d\sigma_f (z).
\end{equation}
Note that for every $z \in \mathbb{T}$ we have $\frac{1}{N} \sum \limits_{n=0}^{N-1}  \left(\frac{z}{\lambda}\right)^n \rightarrow \mathds{1}_{\{\lambda\}}(z)$, hence $| \frac{1}{N} \sum \limits_{n=0}^{N-1}  \left(\frac{z}{\lambda}\right)^n |^2 \rightarrow | \mathds{1}_{\{\lambda\}}(z) |^2 = \mathds{1}_{\{\lambda\}}(z)$. Since $| \frac{1}{N} \sum \limits_{n=0}^{N-1}  \left(\frac{z}{\lambda}\right)^n |^2 \leq \left( \frac{1}{N} \sum \limits_{n=0}^{N-1}  |\frac{z}{\lambda}|^n \right)^2 = 1$, we can make use of Lebesgue's Dominated Convergence Theorem and obtain
\begin{equation}\label{e:measure}
 \int \limits_{\mathbb{T}}  | \frac{1}{N} \sum \limits_{n=0}^{N-1}  \left(\frac{z}{\lambda}\right)^n |^2 d\sigma_f (z) \longrightarrow \int \limits_{\mathbb{T}} \mathds{1}_{\{\lambda\}}(z) d\sigma_f (z) = \sigma_f (\{ \lambda \}).
\end{equation}
Putting together \eqref{e:norm}, \eqref{e:int} and \eqref{e:measure} finishes the proof. $\square$
\end{pf}
Note that this lemma connects notions of spectral measure and eigenfunctions.









\chapter{Wiener-Wintner theorems for deterministic transformations}\label{r:det}

why take $f$ only from $\mathcal{K}^{\bot}$ in Bourgain\\

In this chapter we introduce and prove pointwise Wiener-Wintner type theorems. We start with stating classical Wiener-Wintner theorem, which is a modification of Birkhoff's Ergodic Theorem. It was originally stated by Wiener and Wintner in 1941 (\cite{ww}).

\begin{thm}[Wiener-Wintner ergodic theorem, {\cite[thm. 2.3]{assani}}]\label{t:ww}
Let $(X, \mathcal{A}, \mu, T)$ be an ergodic dynamical system and fix function $f \in \mathscr{L}^1(\mu)$. There exists a measurable set $X_f$ of full measure $(\mu(X_f)=1)$ such that for each $x \in X_f$ the averages
\begin{equation} \label{ww}
\frac{1}{N} \sum \limits_{n=0}^{N-1} \lambda^n f(T^n x)
\end{equation}
converge for all $\lambda \in \mathbb{T}$.
\end{thm}

\begin{rem}
Note that for a fixed $\lambda \in \mathbb{T}$ it is easy to achieve a.e. converengce in \eqref{ww}. Take a product system $(X \times \mathbb{T}, \mathcal{A} \otimes \mathcal{B}(\mathbb{T}), \mu \otimes m, T \times R_{\lambda})$ and observe that it is measure preserving since both $(X, \mathcal{A}, \mu, T)$ and $(\mathbb{T}, \mathcal{B}(\mathbb{T}), m, R_{\lambda})$ are measure preserving. Define a function $g:X \times \mathbb{T} \rightarrow \mathbb{C}$ by $g(x, \omega)=\omega f(x)$. We have $g \in \mathscr{L}^1(\mu \otimes m)$ since, by Fubini's Theorem,
\[ \int \limits_{X \times \mathbb{T}} |g(x, \omega)| d\mu \otimes m (x, \omega) = \int \limits_{X \times \mathbb{T}} |\omega||f(x)| d\mu \otimes m (x, \omega) = \int \limits_{X \times \mathbb{T}} |f(x)| d\mu \otimes m (x, \omega) = \]
\[ =  \int \limits_{X} |f(x)| d\mu(x) < \infty.\]
By Birkhoff's Ergodic Theorem the averages
\[ \frac{1}{N} \sum \limits_{n=0}^{N-1} g(T^n x, R_{\lambda}^n \omega) =\frac{1}{N} \sum \limits_{n=0}^{N-1} g(T^n x, \lambda^n \omega) = \frac{1}{N} \sum \limits_{n=0}^{N-1} \omega \lambda^n f(T^n x)  \]
converge for $\mu \otimes m$ almost all pairs $(x, \omega)$ and (since $\omega \neq 0$) also
\[ \frac{1}{N} \sum \limits_{n=0}^{N-1} \lambda^n f(T^n x) \]
converge $\mu \otimes m$ a.e. The last limit is independent from $\omega$, so this implies $\mu$ a.e. convergence of sequence \eqref{ww}. Further, for a countable subset $C \subset \mathbb{T}$, we can find a set $X_f$ such that \eqref{ww} is convergent for all $x \in X_f$ and $\lambda \in C$ (it is enough to take for $X_f$ an intersection of countably many sets of full measure on which we have convergence for fixed $\lambda \in C$). This shows that the difficulty in Wiener-Wintner theorem is obtaining a set of full measure on which convergence will hold for all (uncountably many) $\lambda \in \mathbb{T}$.
\end{rem}
Three proofs of this theorem can be found in \cite{assani}. We present one of them, which main ingredient is itself a generalization of Wiener-Wintner theorem - its uniform version due to J. Bourgain.

\section{Bourgain's uniform Wiener-Wintner theorem}
In order to state the theorem, we need to introduce the notion of Kronecker factor.

\begin{df}[Kronecker factor]$\\$
Let $(X, \mathcal{A}, \mu, T)$ be a measure preserving system and let $U_T : L^2(\mu) \rightarrow L^2(\mu)$ be its Koopman operator on $L^2(\mu)$. \textbf{Kronecker factor} $\mathcal{K} \subset L^2(\mu)$ is a closure (in $L^2(\mu)$) of a linear subspace spanned by eigenfunctions of $U_T$, i.e.
\[\mathcal{K} := \overline{\text{span}} \left\{ f \in L^2(\mu): f \circ T = \lambda f \text{ for some }\lambda \in \mathbb{C} \right\}.\]
\end{df}

\begin{thm}[Bourgain's uniform Wiener-Wintner theorem {\cite[thm. 2.4]{assani}}]\label{t:bourgain}
Let $(X, \mathcal{A}, \mu, T)$ be an ergodic dynamical system and $f \in \mathcal{K}^{\bot}$. Then for $\mu$ a.e. $x \in X$ we have
\[ \lim \limits_{N \rightarrow \infty}\  \sup \limits_{\lambda \in \mathbb{T}}\ \left| \frac{1}{N} \sum \limits_{n=0}^{N-1} \lambda^n f(T^n x) \right| = 0. \]
\end{thm}
\noindent For the proof of this theorem we need two following lemma's:
\begin{lem}[{\cite[prop. 2.2]{assani}}]
Let $(X, \mathcal{A}, \mu, T)$ be a measure preserving dynamical system. A function $f \in L^2(\mu)$ belongs to $\mathcal{K}^{\bot}$ if and only if its spectral measure $\sigma_f$ is continous.
\end{lem}
\begin{pf} Fix $f \in \mathcal{K}^{\bot}$. Since for every $\lambda \in \mathbb{T}$ for its eigenspace $H_{\lambda}$ we have $H_{\lambda} \subset \mathcal{K}$ and $f$ is orthogonal to $\mathcal{K}$, $f$ must be also orthogonal to $H_{\lambda}$. If $P_{\lambda}$ is an orthogonal projection to $H_{\lambda}$, then we have $P_{\lambda} f =0$. By Lemma \ref{l:spectr} we have $\sigma_f(\{ \lambda \}) = \| P_{\lambda} f \|^2$ for all $\lambda \in \mathbb{T}$, so $\sigma_f (\{ \lambda \})=0$ for all $\lambda \in \mathbb{T}$ and the measure $\sigma_f$ is continuous. Conversly, fix $f \in L^2 (\mu)$ and assume that $\sigma_f$ is continuous. Then again by Lemma \ref{l:spectr} we have $\| P_{\lambda} f \| =0$, hence $f \in H_{\lambda}^{\bot}$ for every $\lambda \in \mathbb{T}$, so $f$ is orthogonal to every eigenfunction of the operator $U_T$. We have (by linearity of the inner product) that $f$ is orthogonal also to $\text{span}\left\{ f \in L^2(\mu): f \circ T = \lambda f \text{ for some }\lambda \in \mathbb{C} \right\}$ and finally (by continuity of the inner product) $f \in \mathcal{K}^{\bot}$. $\square$
\end{pf}

\begin{lem}[Van der Corput inequality, {\cite[thm. 1.7.1]{weber}}]
Let $H$ be a complex Hilbert space. For every finite sequence $x_0, x_1, ..., x_{N-1} \in H$ and integer $R \in \{0,1,...,N-1\}$ the following inequality holds:
\begin{multline*}
\| \frac{1}{N} \sum \limits_{n=0}^{N-1} x_n \|^2 \leq \\
\leq \frac{N+R}{N(R+1)}\left( \frac{1}{N} \sum \limits_{n=0}^{N-1} \|x_n \|^2 + \frac{1}{N(R+1)}\sum \limits_{c=1}^{R} (R-c+1) \sum \limits_{j=0}^{N-c-1} (\langle x_{j+c}, x_j \rangle + \langle x_j, x_{j+c} \rangle) \right).
\end{multline*}
If $H=\mathbb{C}$, this ineqaulity becomes
\begin{equation*}
| \frac{1}{N} \sum \limits_{n=0}^{N-1} x_n |^2 \leq \frac{N+R}{N(R+1)}\left( \frac{1}{N} \sum \limits_{n=0}^{N-1} |x_n |^2 + \frac{2}{N(R+1)}\sum \limits_{c=1}^{R} (R-c+1) \Re \left( \sum \limits_{j=0}^{N-c-1} x_j \overline{x_{j+c}} \right) \right).
\end{equation*}
\end{lem}

\chapter{Ergodic theory for operators}\label{r:ergop}

\chapter{Wiener-Wintner theorem for operator semigroups}\label{r:wwop}

\begin{thebibliography}{99}
\addcontentsline{toc}{chapter}{Bibliography}

\bibitem[Rudin]{rudin} Walter Rudin, \textit{Analiza funkcjonalna}, Wydawnictwo Naukowe PWN, 20??

\bibitem[Assani]{assani} I. Assani,  \textit{Wiener Wintner Ergodic Theorems}, World Scientific, River Edge (2003)

\bibitem[Weber]{weber} M. Weber, \textit{Dynamical Systems and Processes}, EMS Publishing House (2009)

\bibitem[Bellow]{bellow} A. Bellow, V. Losert, \textit{The weighted pointwise ergodic theorem and the individual ergodic theorem along subsequences}, TAMS, vol. 288, no 1 (1985), p. 307 - 345.

\bibitem[Schreiber14]{schreiber14} M. Schreiber, \textit{Topological Wiener - Wintner theorems for amenable operator semigroups}, ETDS  (2014).

\bibitem[Schreiber13]{schreiber13} M. Schreiber, \textit{Uniform families of ergodic operator nets},  Semigroup Forum (2013)

\bibitem[WW]{ww} N. Wiener, A. Wintner, \textit{Harmonic Analysis and Ergodic Theory}, American Journal of Mathematics, Vol. 63, No. 2 (Apr., 1941), pp. 415-426

\bibitem[Eisner et al]{eisner} T. Eisner, B. Farkas, M. Haase, R. Nagel, \textit{Operator Theoretic Aspects of
Ergodic Theory}, Graduate Texts in Mathematics, Springer, to appear, \url{http://www.fan.uni-wuppertal.de/fileadmin/mathe/reine_mathematik/funktionalanalysis/farkas/GTM-master-v82.pdf}

\bibitem[Lema�czyk]{lemanczyk} M. Lema�czyk, \textit{Teoria spektralna dla ergodyk�w}, 2010, script, \url{http://www-users.mat.umk.pl/~mlem/files/Teoria_spektralna_dla_ergodykow.pdf}
\end{thebibliography}
\end{document}